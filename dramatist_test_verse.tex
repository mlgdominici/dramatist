\documentclass[a5paper,11pt]{book}

\usepackage[utf8]{inputenc}
\usepackage[T1]{fontenc}
\usepackage{xjunicode}
\usepackage{dramatist}
% \usepackage{verse}
\usepackage{fancyhdr}
\usepackage{poemscol}
\usepackage{geometry}

\def\testodicontrollo{%
Questo re di Sparta ebbe con voi comune la morte, per giudizio
iniquo degli efori; come voi, per quello d'un ingiusto parlamento.
Ma quanto fu simile l'effetto, altrettanto diversa n'era la
cagione. Agide, col ristabilire l'uguaglianza e la libertá, volea
restituire a Sparta le sue virtú, e il suo splendore; quindi egli
pieno di gloria moriva, eterna di se lasciando la fama. Voi, col
tentare di rompere ogni limite all'autoritá vostra, falsamente il
privato vostro bene procacciarvi bramaste: nulla quindi rimane di
voi; e la sola inutile altrui compassione vi accompagnò nella
tomba.
}

\DRSetup{versedrama}{drama}{leftmargin=1em,
  speakswidth=1em
}

\parindent=1em

% \tracingpatches

\begin{document}

\testodicontrollo

\noindent\rule{\leftmargini}{1pt}

\begin{versedrama}
  \speaker{anfar.}
  Pur, per quanto sia giusto in te lo sdegno,\\
  premilo in petto, se sbramarlo or vuoi.\\
  Io men di te non odio Agide altero;\\
  e la sua pompa di virtudi antiche,\\
  finta in biasmo di noi. Sparta ridurre\\
  qual giá la fea Licurgo, è al par crudele,\\
  che ambizíosa stolidezza: è tale\\
\end{versedrama}

\leftmargini=1em

\begin{verse}
  pure il disegno suo; quindi ebbe ei quasi\\
  la cittá nostra all'ultimo ridotta:\\
  e, sconvolta pur anco, in risse e affanni\\
  egra ella sta. Ma, van cangiando i tempi:\\
  quei traditori, efori allor, che schiavi\\
  eran d'Agesiláo, piú a lui venduti\\
  che ad Agide, con esso ora sbanditi\\
  son tutti, o spenti: e sta in noi soli Sparta.\\
  Ma il popol rio, mendico, e ognor di nuove\\
  cose voglioso, Agide ancora elegge\\
  mezzo a sue mire ingiuste. A schietta forza,\\
  mal frenare il potremmo; ogni novello\\
  governo erra adoprandola. Deluso,\\
  pria che sforzato, il popol sia. Tal cura,\\
  che a cor mi sta non men che a te, mi lascia.\\
  Ecco la madre d'Agide: gran donna\\
  ogni dí piú degli Spartani in core\\
  si fa costei: temer si debbe anch'ella.\\
\end{verse}

\end{document}

